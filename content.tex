%!TEX root = problems.tex

\noindent
In this problem sheet you will create a simple web application that reverses user inputted text. 
The following problems should be completed using the programming language Go~\cite{gowebsite}.
It is recommended that you use the Go package Macaron~\cite{macaronwebsite} and the CSS framework Bootstrap~\cite{bootstrapwebsite}


\begin{questions}


\question
Create a git repository for the following problems.
Make sure your repository has a readme file, a license file and a gitignore file. 


\question
Setup your environment so that you can create a web application.
It is recommended that you install a Go web application framework.


\question
Create a basic web application.
Add a route to serve a resource which simply contains the text ``Hello, world!''.
Test your application so far.


\question
Add a static HTML page resource to your web application.
Have it served as the root resource at \mintinline{html}{/}.
Include any necessary CSS and JS files.


\question
Add a textbox and a label to the HTML page.
Use JavaScript to have the reverse of the string in the text box output to the label.
The label should update every time the user changes the text in the text box.


\question
Add a route using the \mintinline{html}{GET} method to the web server at \mintinline{html}{/reverse}.
This route should accept a single variable, and it should respond with the variable's value reversed as the resource.


\question
Change the JavaScript to use the server-side string reversal, instead of the client-side. 


\question
Change the \mintinline{html}{/reverse} route to use the \mintinline{html}{POST} method instead, and update the JavaScript accordingly.


\end{questions}
